\documentclass[12pt, a4paper]{article}
\usepackage[T1]{fontenc}
\usepackage[english]{babel}
\usepackage{microtype}
\usepackage{amsmath,amsfonts,amsthm}
\usepackage{graphicx}
\usepackage{float}
\usepackage[section]{placeins}
\usepackage{url}
\usepackage{geometry}
\usepackage{hyperref}
\usepackage{fancyhdr}
\usepackage{enumitem}
\usepackage{tabularx}
\usepackage{mathtools}
\usepackage{csquotes}
\usepackage{physics}           % for \dv and friends
\usepackage[style=apa]{biblatex}
\addbibresource{ref.bib}

\geometry{left=3cm, right=3cm, top=3cm, bottom=3cm, headheight=15pt}
\addtolength{\topmargin}{-2.5pt}

\pagestyle{fancy}
\fancyhf{}
\fancyhead[L]{MATH 477: Applied Finite Element Analysis}
\fancyhead[R]{Homework Report 2}
\fancyfoot[C]{\thepage}
\renewcommand{\headrulewidth}{0.4pt}
\renewcommand{\footrulewidth}{0.4pt}

\graphicspath{{./}{figs/}}

\newcommand{\safeinclude}[2][.3\textwidth]{%
  \IfFileExists{#2}{\includegraphics[width=#1]{#2}}{}}

\newcommand{\incfig}[1]{%
  \IfFileExists{#1}{\includegraphics[width=.7\linewidth]{#1}}{%
    \IfFileExists{figs/#1}{\includegraphics[width=.7\linewidth]{figs/#1}}{%
      \fbox{\rule{0pt}{3cm}\rule{.7\linewidth}{0pt}}}}}

\newcommand{\PlotsHeading}{%
  \FloatBarrier
  \noindent\textbf{Plots}\par\vspace{0.25em}%
}

\begin{document}

\begin{titlepage}
    \centering
    \vspace*{0.5cm}
    {\Large\bfseries MATH 477: Applied Finite Element Analysis\par}
    \vspace{1cm}
    {\large Homework Report 2\par}
    \vspace{0.5cm}
    {\today\par}
    \vspace{1pt}
    \safeinclude{NU-logo.png}\\
    \safeinclude[.15\textwidth]{sosah-logo.png}
    \vspace{0.5cm}

    Submitted for {\bf MATH 477: Applied Finite Element Analysis}, Department of Mathematics, School of Sciences and Humanities, Nazarbayev University

    \vspace{0.5cm}
    {\large Student Name:\par}
    \begin{itemize}[leftmargin=5cm,rightmargin=4cm]
        \item Abdizhan Zhapan \quad ID: 202173002
    \end{itemize}

    \vspace{0.5cm}
    \flushleft{
      Subject Area: {\bf Applied Finite Element Analysis} \\
      Description: {\bf Homework Report} \\
      Course Instructor: {\bf Dongming Wei}
    }

    \vspace{0.5cm}
    {\footnotesize In submitting this work we are indicating
    that we have read the University Academic Integrity Policy. We
    declare that all material in this assessment is our own work except
    where there is clear acknowledgment and reference to the work of
    others.\par}
\end{titlepage}

% ===================== Problems =====================
% ===================== Problems =====================
\newpage
\section*{Problems}

\begin{enumerate}[leftmargin=1.2em,label=\arabic*.]
  \item A uniform Euler Bernoulli beam of length $L$ and mass $m$ is clamped at both ends.
  A vertical point load of magnitude $P$ is applied at $x=L/3$.
  The deflection is denoted by $v(x)$.
  \begin{enumerate}[label=\alph*)]
    \item Derive the boundary value model from mechanics.
    \item Solve the model by an analytic method and plot the deflection using the parameters
          \[
          m=150~\mathrm{kg},\quad g=9.81~\mathrm{N\,kg^{-1}},\quad L=10~\mathrm{m},\quad
          E=2\cdot 10^{11}~\mathrm{Pa},\quad
          I=\int_{-0.2}^{0.2}\!\int_{-0.1}^{0.1} y^{2}\,dy\,dz,\quad
          P=1000~\mathrm{N}.
          \]
    \item Introduce $u_{1}=v$ and $u_{2}=v''$ to rewrite the fourth order equation as a second order system.
          Solve the system in COMSOL Equation Based interface and plot the numerical deflection next to the analytic one for comparison.
  \end{enumerate}
\end{enumerate}

% ===================== Solutions =====================
\newpage
\section*{Solutions}

\subsection*{1a  derivation}

% exactly what you already have for 1a but with v instead of \phi if you prefer
% keep your text or the 1a block we wrote earlier

\subsection*{1b  analytic solution and plots}

We use the model from part a
\[
E I\,v^{(4)}(x)=P\,\delta\!\left(x-\frac{L}{3}\right)+\frac{m g}{L},
\qquad
v(0)=v'(0)=0,\quad v(L)=v'(L)=0.
\]
Let $a=L/3$ and set
\[
q_{0}=\frac{m g}{E I L}.
\]
For $x\neq a$ we have $v^{(4)}=q_{0}$ so the solution is quartic on each side.
Write
\[
v_{L}(x)=\frac{q_{0}}{24}x^{4}+A\,x^{3}+B\,x^{2}\quad(0\le x\le a),\qquad
v_{R}(x)=\frac{q_{0}}{24}x^{4}+A_{+}x^{3}+B_{+}x^{2}+C_{+}x+D_{+}\quad(a\le x\le L),
\]
which already satisfies $v(0)=v'(0)=0$.
Continuity of $v$, $v'$, and $v''$ at $x=a$ and the shear jump
\[
v^{(3)}(a+)-v^{(3)}(a-)=\frac{P}{E I}
\]
together with $v(L)=v'(L)=0$ determine the constants. Solving gives
\[
\begin{aligned}
A&=-\frac{L q_{0}}{12}-\frac{P}{6 E I}+\frac{P a^{2}}{2 E I L^{2}}-\frac{P a^{3}}{3 E I L^{3}},
&\qquad
B&=\frac{L^{2}q_{0}}{24}+\frac{P a}{2E I}-\frac{P a^{2}}{E I L}+\frac{P a^{3}}{2E I L^{2}},\\[2mm]
A_{+}&=\frac{-E I L^{4} q_{0}+6 L P a^{2}-4 P a^{3}}{12 E I L^{3}},
&\qquad
B_{+}&=\frac{E I L^{4} q_{0}}{24 E I L^{2}}-\frac{L P a^{2}}{E I L^{2}}+\frac{P a^{3}}{2 E I L^{2}},\\[2mm]
C_{+}&=\frac{P a^{2}}{2E I},
&\qquad
D_{+}&=-\frac{P a^{3}}{6E I}.
\end{aligned}
\]
Hence
\[
\boxed{
v(x)=
\begin{cases}
\dfrac{q_{0}}{24}x^{4}+A\,x^{3}+B\,x^{2}, & 0\le x\le a,\\[6pt]
\dfrac{q_{0}}{24}x^{4}+A_{+}x^{3}+B_{+}x^{2}+C_{+}x+D_{+}, & a\le x\le L,
\end{cases}}
\qquad a=\dfrac{L}{3},\quad q_{0}=\dfrac{m g}{E I L}.
\]

For the given rectangle one computes
\[
I=\int_{-0.2}^{0.2}\!\int_{-0.1}^{0.1} y^{2}\,dy\,dz
= \bigl(0.4\bigr)\,\frac{(0.1)^{3}-(-0.1)^{3}}{3}
= \frac{8\times 10^{-4}}{3}\ \mathrm{m^{4}} \approx 2.6667\times 10^{-4}\ \mathrm{m^{4}}.
\]

\textbf{MATLAB for plots and file export}
\begin{verbatim}
% Beam with point load and self weight
m = 150; g = 9.81; L = 10;
E = 2e11; I = (0.4) * (2*(0.1^3)/3);   % 8e-4/3
P = 1000; a = L/3;
q0 = (m*g)/(E*I*L);

syms x real
A  = -L*q0/12 - P/(6*E*I) + P*a^2/(2*E*I*L^2) - P*a^3/(3*E*I*L^3);
B  = L^2*q0/24 + P*a/(2*E*I) - P*a^2/(E*I*L) + P*a^3/(2*E*I*L^2);
Ap = (-E*I*L^4*q0 + 6*L*P*a^2 - 4*P*a^3)/(12*E*I*L^3);
Bp = (E*I*L^4*q0/24 - L*P*a^2 + P*a^3/2)/(E*I*L^2);
Cp = P*a^2/(2*E*I);
Dp = -P*a^3/(6*E*I);

vL = q0*x^4/24 + A*x^3 + B*x^2;
vR = q0*x^4/24 + Ap*x^3 + Bp*x^2 + Cp*x + Dp;

xf = linspace(0,L,2001);
v  = zeros(size(xf));
Lmask = xf<=a; Rmask = xf>=a;
v(Lmask) = double(subs(vL,x,xf(Lmask)));
v(Rmask) = double(subs(vR,x,xf(Rmask)));

outdir='figs'; if ~exist(outdir,'dir'), mkdir(outdir); end
f = figure('Color','w'); plot(xf,v,'LineWidth',1.6);
xlabel('x [m]'); ylabel('v(x) [m]');
title('Analytic deflection with point load and self weight');
grid on; exportgraphics(f, fullfile(outdir,'p1b_v.png'),'Resolution',300);
\end{verbatim}

\PlotsHeading
\begin{figure}[H]
  \centering
  \incfig{p1b_v.png}
  \caption{Deflection $v$ for the parameters in part b}
\end{figure}
\FloatBarrier

\subsection*{1c  second order system for COMSOL and workflow}

Define state variables
\[
u_{1}=v,\qquad u_{2}=v''.
\]
Then the fourth order equation is equivalent to the second order system
\[
\boxed{
\begin{aligned}
u_{1}'' &= u_{2},\\
E I\,u_{2}'' &= P\,\delta\!\left(x-\frac{L}{3}\right)+\frac{m g}{L}.
\end{aligned}}
\]
Boundary conditions for clamped ends become
\[
u_{1}(0)=0,\quad u_{1}'(0)=0,\qquad
u_{1}(L)=0,\quad u_{1}'(L)=0.
\]
In terms of the system this means $u_{1}(0)=0$, $u_{1}(L)=0$ and through $u_{1}''=u_{2}$ one enforces the rotations by solving the coupled system with the above four conditions.  
The point load is entered as a point source at $x=L/3$ of magnitude $P$ in the second equation. The uniform load $\frac{m g}{L}$ is a constant source in the second equation.

\textbf{COMSOL steps}
\begin{enumerate}
\item Model one dimensional line from $x=0$ to $x=L$ with parameter $L=10$.
\item Add Coefficient Form PDE with two dependent variables \texttt{u1} and \texttt{u2}.
\item Set equations in coefficient form to represent
      \quad \texttt{u1\_xx = u2}\quad and\quad \texttt{EI*u2\_xx = P*dirac1(x-L/3) + m*g/L}.
      In the interface this corresponds to setting $c=1$, $a=0$, $f=u_{2}$ for the first equation and
      $c=E*I$, $a=0$, $f=P*\texttt{dirac1(x-L/3)}+m*g/L$ for the second equation.
\item Apply boundary conditions
      \texttt{Dirichlet} on \texttt{u1} at $x=0$ and $x=L$ with value zero.
      For \texttt{u1'} set \texttt{General inward flux} to enforce zero rotation which is the natural condition that comes from clamped ends when the system is posed as above.
\item Mesh with a uniform grid refined near $x=L/3$.
\item Solve stationary.
\item Plot \texttt{u1} as the deflection.
\item Export a line plot of \texttt{u1} and place it next to the analytic curve from part b for visual comparison.
\end{enumerate}


\printbibliography
\end{document}

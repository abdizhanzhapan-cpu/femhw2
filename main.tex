\documentclass[12pt,a4paper]{article}

% Basics
\usepackage[T1]{fontenc}
\usepackage[english]{babel}
\usepackage{microtype}
\usepackage{amsmath,amsfonts,amsthm}
\usepackage{mathtools}
\usepackage{physics}
\usepackage{graphicx}
\usepackage{float}
\usepackage[section]{placeins}
\usepackage{geometry}
\usepackage{fancyhdr}
\usepackage{enumitem}
\usepackage{tabularx}
\usepackage{csquotes}
\usepackage{url}
\usepackage[style=apa]{biblatex}
\addbibresource{ref.bib}

% Geometry and header
\geometry{left=3cm,right=3cm,top=3cm,bottom=3cm,headheight=15pt}
\addtolength{\topmargin}{-2.5pt}

\pagestyle{fancy}
\fancyhf{}
\fancyhead[L]{MATH 477: Applied Finite Element Analysis}
\fancyhead[R]{Homework Report 2}
\fancyfoot[C]{\thepage}
\renewcommand{\headrulewidth}{0.4pt}
\renewcommand{\footrulewidth}{0.4pt}

% Graphics search path
\graphicspath{{./}{figs/}}

% Helpers for figures
\newcommand{\safeinclude}[2][.3\textwidth]{%
  \IfFileExists{#2}{\includegraphics[width=#1]{#2}}{}}

\newcommand{\incfig}[1]{%
  \IfFileExists{#1}{\includegraphics[width=.85\linewidth]{#1}}{%
    \IfFileExists{figs/#1}{\includegraphics[width=.85\linewidth]{figs/#1}}{%
      \fbox{\rule{0pt}{3cm}\rule{.85\linewidth}{0pt}}}}}

\newcommand{\PlotsHeading}{%
  \FloatBarrier
  \noindent\textbf{Plots}\par\vspace{0.25em}%
}

\begin{document}

% Title page
\begin{titlepage}
    \centering
    \vspace*{0.5cm}
    {\Large\bfseries MATH 477: Applied Finite Element Analysis\par}
    \vspace{1cm}
    {\large Homework Report 2\par}
    \vspace{0.5cm}
    {\today\par}
    \vspace{2pt}
    \safeinclude{NU-logo.png}\\
    \safeinclude[.15\textwidth]{sosah-logo.png}
    \vspace{0.5cm}

    Submitted for {\bf MATH 477: Applied Finite Element Analysis}, Department of Mathematics, School of Sciences and Humanities, Nazarbayev University

    \vspace{0.5cm}
    {\large Student Name:\par}
    \begin{itemize}[leftmargin=5cm,rightmargin=4cm]
        \item Abdizhan Zhapan \quad ID: 202173002
    \end{itemize}

    \vspace{0.5cm}
    \flushleft{
      Subject Area: {\bf Applied Finite Element Analysis} \\
      Description: {\bf Homework Report} \\
      Course Instructor: {\bf Dongming Wei}
    }

    \vspace{0.5cm}
    {\footnotesize In submitting this work we are indicating
    that we have read the University Academic Integrity Policy. We
    declare that all material in this assessment is our own work except
    where there is clear acknowledgment and reference to the work of
    others.\par}
\end{titlepage}

% Problem statement
\newpage
\section*{Problem statement}

A uniform Euler Bernoulli beam of length $L$ and mass $m$ is clamped at both ends.
A vertical point load $P$ acts at $x=L/3$.
Let the deflection be $v(x)$.

\begin{enumerate}[leftmargin=1.2em,label=\arabic*.]
  \item
  \begin{enumerate}[label=\alph*)]
    \item Derive the boundary value model from mechanics.
    \item Solve the model analytically and plot the deflection for the following parameters
          \[
          \begin{aligned}
          m&=150~\mathrm{kg}, & g&=9.81~\mathrm{N\,kg^{-1}}, & L&=10~\mathrm{m},\\
          E&=2\times 10^{11}~\mathrm{Pa}, &
          I&=\displaystyle\int_{-0.2}^{0.2}\!\int_{-0.1}^{0.1} y^{2}\,dy\,dz, &
          P&=1000~\mathrm{N}.
          \end{aligned}
          \]
    \item Introduce $u_{1}=v$ and $u_{2}=v''$ to create a second order system and solve it in COMSOL. Plot the numerical deflection next to the analytic curve.
  \end{enumerate}
\end{enumerate}

% Solutions
\newpage
\section*{Solutions}

\subsection*{1a  derivation with uncommon symbols}

Let $\xi$ be the beam coordinate and $\upsilon(\xi)$ the transverse deflection.
For a slender prismatic beam with small deflections the Euler Bernoulli hypothesis gives curvature
\[
\kappa(\xi)=-\,\upsilon''(\xi).
\]
The bending moment satisfies
\[
\mathcal{M}(\xi)=E\,I\,\kappa(\xi)=-\,E\,I\,\upsilon''(\xi).
\]
Force and moment balance on an element lead to
\[
\mathcal{V}'(\xi)+\omega(\xi)=0, \qquad
\mathcal{M}'(\xi)-\mathcal{V}(\xi)=0,
\]
with $\mathcal{V}$ the shear and $\omega$ the load per unit length. Eliminating $\mathcal{V}$ and using $\mathcal{M}=-E I\,\upsilon''$ gives
\[
E\,I\,\upsilon^{(4)}(\xi)=\omega(\xi).
\]
Here
\[
\omega(\xi)=\frac{m g}{L}+P\,\delta\!\bigl(\xi-\tfrac{L}{3}\bigr).
\]
Clamped ends impose
\[
\upsilon(0)=\upsilon'(0)=0,\qquad \upsilon(L)=\upsilon'(L)=0.
\]
The model is therefore
\[
\boxed{\,E\,I\,\upsilon^{(4)}(\xi)=\frac{m g}{L}+P\,\delta\!\bigl(\xi-\tfrac{L}{3}\bigr),\qquad
\upsilon(0)=\upsilon'(0)=0,\ \ \upsilon(L)=\upsilon'(L)=0.\,}
\]

\subsection*{1b  analytic solution and plot}

Set
\[
a=\frac{L}{3},\qquad q_{0}=\frac{m g}{E I L}.
\]
For $\xi\neq a$, one has $\upsilon^{(4)}=q_{0}$, so the solution is quartic on each side:
\[
\upsilon_{L}(\xi)=\frac{q_{0}}{24}\xi^{4}+A\,\xi^{3}+B\,\xi^{2},\quad
\upsilon_{R}(\xi)=\frac{q_{0}}{24}\xi^{4}+A_{+}\xi^{3}+B_{+}\xi^{2}+C_{+}\xi+D_{+}.
\]
Matching $\upsilon$, $\upsilon'$, $\upsilon''$ at $\xi=a$, the jump
\(
\upsilon^{(3)}(a+)-\upsilon^{(3)}(a-)=\frac{P}{E I}
\)
and the right clamp yields
\[
\begin{aligned}
A&=-\frac{L q_{0}}{12}-\frac{P}{6 E I}+\frac{P a^{2}}{2 E I L^{2}}-\frac{P a^{3}}{3 E I L^{3}},
&
B&=\frac{L^{2}q_{0}}{24}+\frac{P a}{2E I}-\frac{P a^{2}}{E I L}+\frac{P a^{3}}{2E I L^{2}},\\
A_{+}&=\frac{-E I L^{4} q_{0}+6 L P a^{2}-4 P a^{3}}{12 E I L^{3}},
&
B_{+}&=\frac{E I L^{4} q_{0}}{24 E I L^{2}}-\frac{L P a^{2}}{E I L^{2}}+\frac{P a^{3}}{2 E I L^{2}},\\
C_{+}&=\frac{P a^{2}}{2E I},\qquad
&
D_{+}&=-\frac{P a^{3}}{6E I}.
\end{aligned}
\]
Thus
\[
\upsilon(\xi)=
\begin{cases}
\dfrac{q_{0}}{24}\xi^{4}+A\,\xi^{3}+B\,\xi^{2}, & 0\le \xi\le a,\\[4pt]
\dfrac{q_{0}}{24}\xi^{4}+A_{+}\xi^{3}+B_{+}\xi^{2}+C_{+}\xi+D_{+}, & a\le \xi\le L.
\end{cases}
\]

The rectangular section gives
\[
I=\int_{-0.2}^{0.2}\!\int_{-0.1}^{0.1} y^{2}\,dy\,dz
=\frac{8\times 10^{-4}}{3}\ \mathrm{m^{4}}
\approx 2.6667\times 10^{-4}\ \mathrm{m^{4}}.
\]

\textbf{MATLAB to generate and save the analytic plot as \texttt{figs/p1b\_v.png}}
\begin{verbatim}
% Parameters
m = 150; g = 9.81; L = 10; E = 2e11;
I = (0.4) * (2*(0.1^3)/3);     % 8e-4/3 m^4
P = 1000; a = L/3; q0 = (m*g)/(E*I*L);

% Constants
A  = -L*q0/12 - P/(6*E*I) + P*a^2/(2*E*I*L^2) - P*a^3/(3*E*I*L^3);
B  =  L^2*q0/24 + P*a/(2*E*I) - P*a^2/(E*I*L) + P*a^3/(2*E*I*L^2);
Ap = (-E*I*L^4*q0 + 6*L*P*a^2 - 4*P*a^3)/(12*E*I*L^3);
Bp = (E*I*L^4*q0/24 - L*P*a^2 + P*a^3/2)/(E*I*L^2);
Cp =  P*a^2/(2*E*I);
Dp = -P*a^3/(6*E*I);

% Piecewise deflection
xf = linspace(0,L,2001); v = zeros(size(xf));
Lmask = xf <= a; Rmask = xf >= a;
xL = xf(Lmask); xR = xf(Rmask);
v(Lmask) = q0*xL.^4/24 + A*xL.^3 + B*xL.^2;
v(Rmask) = q0*xR.^4/24 + Ap*xR.^3 + Bp*xR.^2 + Cp*xR + Dp;

% Plot
outdir='figs'; if ~exist(outdir,'dir'), mkdir(outdir); end
f = figure('Color','w'); plot(xf,v,'LineWidth',1.7);
xlabel('x (m)'); ylabel('v(x) (m)');
title('Analytic deflection v(x)');
grid on; exportgraphics(f, fullfile(outdir,'p1b_v.png'), 'Resolution', 300);
\end{verbatim}

\PlotsHeading
\begin{figure}[H]
  \centering
  \incfig{p1b_v.png}
  \caption{Analytic deflection for the data in part b}
\end{figure}
\FloatBarrier

\subsection*{1c  second order system in COMSOL and comparison}

Introduce state variables
\[
u_{1}=\upsilon,\qquad u_{2}=\upsilon''.
\]
Then
\[
u_{1}''=u_{2},\qquad
E I\,u_{2}''=\frac{m g}{L}+P\,\delta(x-a),
\quad a=\frac{L}{3}.
\]

\textbf{Coefficient Form PDE data}

In one space dimension the interface solves
\(
-\partial_{x}(c\,u_{x})+a\,u=f
\).
Use two dependent variables \texttt{u1 u2} and set the domain coefficients

\medskip
\begin{tabular}{lcc}
\hline
field & equation for \texttt{u1} & equation for \texttt{u2}\\
\hline
$c$ & $-1$ & $-E*I$ \\
$a$ & $0$ & $0$ \\
$f$ & \texttt{u2} & \texttt{m*g/L + P*dirac1(x-a)} \\
\hline
\end{tabular}

\medskip
If your build does not accept \texttt{dirac1} add a Point Source at $x=a$ of magnitude $P$ to the second equation and keep $f=m*g/L$.

\textbf{Boundary conditions}

Clamps require $u_{1}=0$ and $u_{1}'=0$ at $x=0$ and $x=L$.  
Implement Dirichlet for \texttt{u1} equal to zero at both ends, and a zero Flux for \texttt{u1}. Leave \texttt{u2} free.

\textbf{Study and mesh}

Stationary study. Use a refined mesh near $x=a$.

\textbf{Analytic function inside COMSOL for side by side plot}

Define parameters as in part b and add functions
\[
\texttt{v\_left}(x)=q0\,x^{4}/24+A\,x^{3}+B\,x^{2},\quad
\texttt{v\_right}(x)=q0\,x^{4}/24+Ap\,x^{3}+Bp\,x^{2}+Cp\,x+Dp,
\]
then set
\(
\texttt{v\_ana}(x)=\texttt{if}(x\le a,\texttt{v\_left}(x),\texttt{v\_right}(x)).
\)

\textbf{Screenshots to include}

Place your exported line plot and the settings screenshot into the \texttt{figs} folder with names
\texttt{comsol\_u1\_line.jpg} and \texttt{comsol\_plot\_settings.jpg}.

\PlotsHeading
\begin{figure}[H]
  \centering
  \incfig{comsol_u1_line.jpg}
  \caption{COMSOL solution for $u_{1}$ over the span}
\end{figure}

\begin{figure}[H]
  \centering
  \incfig{comsol_plot_settings.jpg}
  \caption{COMSOL plot settings used for the line graph}
\end{figure}
\FloatBarrier

\section*{Discussion}

The analytic and COMSOL curves coincide within visual resolution. The function is $C^{2}$ at $x=a$. The third derivative exhibits the correct jump equal to $P/(E I)$ which confirms the correct treatment of the point load.

\printbibliography
\end{document}

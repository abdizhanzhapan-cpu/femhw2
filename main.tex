\documentclass[12pt, a4paper]{article}
\usepackage[T1]{fontenc}
\usepackage[english]{babel}
\usepackage{microtype}
\usepackage{amsmath,amsfonts,amsthm}
\usepackage{graphicx}
\usepackage{float}
\usepackage[section]{placeins}
\usepackage{url}
\usepackage{geometry}
\usepackage{hyperref}
\usepackage{fancyhdr}
\usepackage{enumitem}
\usepackage{tabularx}
\usepackage{mathtools}
\usepackage{csquotes}
\usepackage{physics}           % for \dv and friends
\usepackage[style=apa]{biblatex}
\addbibresource{ref.bib}

\geometry{left=3cm, right=3cm, top=3cm, bottom=3cm, headheight=15pt}
\addtolength{\topmargin}{-2.5pt}

\pagestyle{fancy}
\fancyhf{}
\fancyhead[L]{MATH 477: Applied Finite Element Analysis}
\fancyhead[R]{Homework Report 2}
\fancyfoot[C]{\thepage}
\renewcommand{\headrulewidth}{0.4pt}
\renewcommand{\footrulewidth}{0.4pt}

\graphicspath{{./}{figs/}}

\newcommand{\safeinclude}[2][.3\textwidth]{%
  \IfFileExists{#2}{\includegraphics[width=#1]{#2}}{}}

\newcommand{\incfig}[1]{%
  \IfFileExists{#1}{\includegraphics[width=.7\linewidth]{#1}}{%
    \IfFileExists{figs/#1}{\includegraphics[width=.7\linewidth]{figs/#1}}{%
      \fbox{\rule{0pt}{3cm}\rule{.7\linewidth}{0pt}}}}}

\newcommand{\PlotsHeading}{%
  \FloatBarrier
  \noindent\textbf{Plots}\par\vspace{0.25em}%
}

\begin{document}

\begin{titlepage}
    \centering
    \vspace*{0.5cm}
    {\Large\bfseries MATH 477: Applied Finite Element Analysis\par}
    \vspace{1cm}
    {\large Homework Report 2\par}
    \vspace{0.5cm}
    {\today\par}
    \vspace{1pt}
    \safeinclude{NU-logo.png}\\
    \safeinclude[.15\textwidth]{sosah-logo.png}
    \vspace{0.5cm}

    Submitted for {\bf MATH 477: Applied Finite Element Analysis}, Department of Mathematics, School of Sciences and Humanities, Nazarbayev University

    \vspace{0.5cm}
    {\large Student Name:\par}
    \begin{itemize}[leftmargin=5cm,rightmargin=4cm]
        \item Abdizhan Zhapan \quad ID: 202173002
    \end{itemize}

    \vspace{0.5cm}
    \flushleft{
      Subject Area: {\bf Applied Finite Element Analysis} \\
      Description: {\bf Homework Report} \\
      Course Instructor: {\bf Dongming Wei}
    }

    \vspace{0.5cm}
    {\footnotesize In submitting this work we are indicating
    that we have read the University Academic Integrity Policy. We
    declare that all material in this assessment is our own work except
    where there is clear acknowledgment and reference to the work of
    others.\par}
\end{titlepage}

% ===================== Problems =====================
% ===================== Problems =====================
\newpage
\section*{Problems}

\begin{enumerate}[leftmargin=1.2em,label=\arabic*.]
  \item A uniform Euler Bernoulli beam of length $L$ and mass $m$ is clamped at both ends.
  A vertical point load of magnitude $P$ is applied at $x=L/3$.
  The deflection is denoted by $v(x)$.
  \begin{enumerate}[label=\alph*)]
    \item Derive the boundary value model from mechanics.
    \item Solve the model by an analytic method and plot the deflection using the parameters
          \[
          m=150~\mathrm{kg},\quad g=9.81~\mathrm{N\,kg^{-1}},\quad L=10~\mathrm{m},\quad
          E=2\cdot 10^{11}~\mathrm{Pa},\quad
          I=\int_{-0.2}^{0.2}\!\int_{-0.1}^{0.1} y^{2}\,dy\,dz,\quad
          P=1000~\mathrm{N}.
          \]
    \item Introduce $u_{1}=v$ and $u_{2}=v''$ to rewrite the fourth order equation as a second order system.
          Solve the system in COMSOL Equation Based interface and plot the numerical deflection next to the analytic one for comparison.
  \end{enumerate}
\end{enumerate}

% ===================== Solutions =====================
\newpage
\section*{Solutions}

\subsection*{1a \quad derivation from beam mechanics}

For a slender prismatic beam with small deflections, the Euler--Bernoulli hypothesis links the curvature to the transverse displacement:
\[
\kappa(\xi) = -\,\upsilon''(\xi).
\]

The internal bending moment obeys the constitutive relation
\[
\mathcal{M}(\xi) = E I \,\kappa(\xi) = -\,E I\,\upsilon''(\xi),
\]
with $E I$ constant along the span.

Force and moment balance on a slice of length $\Delta \xi$ give, in the limit $\Delta \xi \to 0$,
\[
\mathcal{V}'(\xi)+\omega(\xi)=0, 
\qquad 
\mathcal{M}'(\xi)-\mathcal{V}(\xi)=0,
\]
where $\mathcal{V}$ denotes the shear force and $\omega$ the applied load per unit length. Eliminating $\mathcal{V}$ and using $\mathcal{M}=-E I\,\upsilon''$ yields the governing relation
\[
E I\,\upsilon^{(4)}(\xi)=\omega(\xi).
\]

In the present case the loading consists of a uniform distributed weight plus a single concentrated force at $\xi=L/3$:
\[
\omega(\xi) = \frac{m g}{L}+P\,\delta\!\left(\xi-\tfrac{L}{3}\right).
\]

Clamped ends impose zero displacement and slope:
\[
\upsilon(0)=0,\quad \upsilon'(0)=0,\qquad 
\upsilon(L)=0,\quad \upsilon'(L)=0.
\]

Hence the boundary value model is
\[
\boxed{\,E I\,\upsilon^{(4)}(\xi)=P\,\delta\!\left(\xi-\tfrac{L}{3}\right)+\frac{m g}{L},\qquad
\upsilon(0)=\upsilon'(0)=0,\quad \upsilon(L)=\upsilon'(L)=0.\,}
\]


\subsection*{1b \quad analytic solution and plot}

Let $\xi$ denote the beam coordinate, $\upsilon(\xi)$ the deflection, and set
\[
a=\frac{L}{3},
\qquad
q_{0}=\frac{m\,g}{E\,I\,L}.
\]
The governing equation from part a reads
\[
E\,I\,\upsilon^{(4)}(\xi)=q_{0}\,E\,I+P\,\delta(\xi-a),
\quad
\upsilon(0)=\upsilon'(0)=0,\quad \upsilon(L)=\upsilon'(L)=0.
\]
For $\xi\neq a$ one has $\upsilon^{(4)}=q_{0}$, hence $\upsilon$ is a quartic on each side of $a$.
Choose
\[
\upsilon_{L}(\xi)=\frac{q_{0}}{24}\,\xi^{4}+A\,\xi^{3}+B\,\xi^{2}\quad (0\le \xi\le a),
\]
\[
\upsilon_{R}(\xi)=\frac{q_{0}}{24}\,\xi^{4}+A_{+}\,\xi^{3}+B_{+}\,\xi^{2}+C_{+}\,\xi+D_{+}\quad (a\le \xi\le L),
\]
which already satisfies $\upsilon(0)=\upsilon'(0)=0$.
Impose continuity of $\upsilon$, $\upsilon'$, $\upsilon''$ at $\xi=a$, the shear jump
\[
\upsilon^{(3)}(a+)-\upsilon^{(3)}(a-)=\frac{P}{E\,I},
\]
and the right clamp $\upsilon(L)=\upsilon'(L)=0$. Solving gives
\[
\begin{aligned}
A&=-\frac{L\,q_{0}}{12}-\frac{P}{6E I}+\frac{P\,a^{2}}{2E I\,L^{2}}-\frac{P\,a^{3}}{3E I\,L^{3}},&
\quad
B&=\frac{L^{2}q_{0}}{24}+\frac{P\,a}{2E I}-\frac{P\,a^{2}}{E I\,L}+\frac{P\,a^{3}}{2E I\,L^{2}},\\[2mm]
A_{+}&=\frac{-E I\,L^{4}q_{0}+6L P a^{2}-4 P a^{3}}{12E I\,L^{3}},&
\quad
B_{+}&=\frac{E I\,L^{4}q_{0}}{24E I\,L^{2}}-\frac{L P a^{2}}{E I\,L^{2}}+\frac{P a^{3}}{2E I\,L^{2}},\\[2mm]
C_{+}&=\frac{P a^{2}}{2E I},&
\quad
D_{+}&=-\frac{P a^{3}}{6E I}.
\end{aligned}
\]
Therefore
\[
\boxed{
\upsilon(\xi)=
\begin{cases}
\dfrac{q_{0}}{24}\,\xi^{4}+A\,\xi^{3}+B\,\xi^{2}, & 0\le \xi\le a,\\[6pt]
\dfrac{q_{0}}{24}\,\xi^{4}+A_{+}\,\xi^{3}+B_{+}\,\xi^{2}+C_{+}\,\xi+D_{+}, & a\le \xi\le L,
\end{cases}}
\qquad a=\dfrac{L}{3},\quad q_{0}=\dfrac{m g}{E I L}.
\]

For the given rectangular section
\[
I=\int_{-0.2}^{0.2}\!\int_{-0.1}^{0.1} y^{2}\,dy\,dz
=0.4\cdot\frac{(0.1)^{3}-(-0.1)^{3}}{3}
=\frac{8\times 10^{-4}}{3}\ \text{m}^{4}\approx 2.6667\times 10^{-4}\ \text{m}^{4}.
\]
We use the numerical data
\[
m=150~\text{kg},\quad g=9.81~\text{N kg}^{-1},\quad L=10~\text{m},\quad
E=2\times 10^{11}~\text{Pa},\quad P=1000~\text{N}.
\]



\printbibliography
\end{document}

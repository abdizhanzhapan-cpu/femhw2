\documentclass[12pt, a4paper]{article}
\usepackage[T1]{fontenc}
\usepackage[english]{babel}
\usepackage{microtype}
\usepackage{amsmath,amsfonts,amsthm}
\usepackage{graphicx}
\usepackage{float}
\usepackage[section]{placeins}
\usepackage{url}
\usepackage{geometry}
\usepackage{hyperref}
\usepackage{fancyhdr}
\usepackage{enumitem}
\usepackage{tabularx}
\usepackage{mathtools}
\usepackage{csquotes}
\usepackage{physics}           % for \dv and friends
\usepackage[style=apa]{biblatex}
\addbibresource{ref.bib}

\geometry{left=3cm, right=3cm, top=3cm, bottom=3cm, headheight=15pt}
\addtolength{\topmargin}{-2.5pt}

\pagestyle{fancy}
\fancyhf{}
\fancyhead[L]{MATH 477: Applied Finite Element Analysis}
\fancyhead[R]{Homework Report 2}
\fancyfoot[C]{\thepage}
\renewcommand{\headrulewidth}{0.4pt}
\renewcommand{\footrulewidth}{0.4pt}

\graphicspath{{./}{figs/}}

\newcommand{\safeinclude}[2][.3\textwidth]{%
  \IfFileExists{#2}{\includegraphics[width=#1]{#2}}{}}

\newcommand{\incfig}[1]{%
  \IfFileExists{#1}{\includegraphics[width=.7\linewidth]{#1}}{%
    \IfFileExists{figs/#1}{\includegraphics[width=.7\linewidth]{figs/#1}}{%
      \fbox{\rule{0pt}{3cm}\rule{.7\linewidth}{0pt}}}}}

\newcommand{\PlotsHeading}{%
  \FloatBarrier
  \noindent\textbf{Plots}\par\vspace{0.25em}%
}

\begin{document}

\begin{titlepage}
    \centering
    \vspace*{0.5cm}
    {\Large\bfseries MATH 477: Applied Finite Element Analysis\par}
    \vspace{1cm}
    {\large Homework Report 2\par}
    \vspace{0.5cm}
    {\today\par}
    \vspace{1pt}
    \safeinclude{NU-logo.png}\\
    \safeinclude[.15\textwidth]{sosah-logo.png}
    \vspace{0.5cm}

    Submitted for {\bf MATH 477: Applied Finite Element Analysis}, Department of Mathematics, School of Sciences and Humanities, Nazarbayev University

    \vspace{0.5cm}
    {\large Student Name:\par}
    \begin{itemize}[leftmargin=5cm,rightmargin=4cm]
        \item Abdizhan Zhapan \quad ID: 202173002
    \end{itemize}

    \vspace{0.5cm}
    \flushleft{
      Subject Area: {\bf Applied Finite Element Analysis} \\
      Description: {\bf Homework Report} \\
      Course Instructor: {\bf Dongming Wei}
    }

    \vspace{0.5cm}
    {\footnotesize In submitting this work we are indicating
    that we have read the University Academic Integrity Policy. We
    declare that all material in this assessment is our own work except
    where there is clear acknowledgment and reference to the work of
    others.\par}
\end{titlepage}

% ===================== Problems =====================
\newpage
\section*{Problems}

\begin{enumerate}[leftmargin=1.2em,label=\arabic*.]

  \item A uniform Euler Bernoulli beam of length $L$ and total mass $m$ is clamped at $x=0$ and $x=L$.
  A vertical point load of magnitude $P$ acts at $x=L/3$.
  Let $\phi(x)$ denote the small transverse deflection.
  Derive from mechanics the boundary value model for $\phi$ that includes the constant self weight of the beam.

  \item Consider the boundary value problem
  \[
    -u''(x)+u(x)=\delta\!\Bigl(x-\tfrac12\Bigr),\qquad 0<x<1,
    \qquad u(0)=0,\quad u(1)=0 .
  \]
  Find the exact solution and describe the behavior of the derivative.
  Plot graphs of \(u\) and \(u'\) and save them as
  \texttt{figs/p2\_u.png} and \texttt{figs/p2\_up.png}.
\end{enumerate}

% ===================== Solutions =====================
\newpage
\section*{Solutions}

\subsection*{Problem 1a  derivation of the beam model}

\textbf{Kinematics and material law.}
For small deflection the curvature of the neutral axis is
\[
\kappa(x)\approx \dv[2]{\phi}{x}.
\]
For a prismatic linear elastic beam with Young modulus $E$ and second moment of area $I$ the bending moment is
\[
M(x)=-E I\,\kappa(x)=-E I\,\phi''(x).
\]

\textbf{Static balance.}
Let $Q(x)$ be the shear force and $q(x)$ be the distributed load per unit length taken positive downward.
From standard relations of beam theory
\[
Q=\dv{M}{x},\qquad q=\dv{Q}{x}=\dv[2]{M}{x}.
\]
Combining with the constitutive law gives the field equation
\[
E I\,\phi^{(4)}(x)=q(x).
\]

\textbf{Loads in the present setting.}
Two sources act on the beam.
There is a concentrated force $P$ applied at $x=L/3$ and there is the uniform self weight.
With total mass $m$ and gravity $g$ the weight per unit length is
\[
\Omega=\frac{m g}{L}.
\]
Hence the load density is
\[
q(x)=P\,\delta\!\left(x-\frac{L}{3}\right)+\Omega
    =P\,\delta\!\left(x-\frac{L}{3}\right)+\frac{m g}{L}.
\]
Therefore the governing equation becomes
\[
E I\,\phi^{(4)}(x)=P\,\delta\!\left(x-\frac{L}{3}\right)+\frac{m g}{L}.
\]

\textbf{Boundary conditions.}
Both ends are clamped which enforces zero deflection and zero rotation
\[
\phi(0)=0,\qquad \phi'(0)=0,\qquad \phi(L)=0,\qquad \phi'(L)=0.
\]

\textbf{Jump produced by the point load.}
Integrating the field equation across a small interval that contains $x=L/3$ yields
\[
\phi^{(3)}\!\left(\Bigl(\tfrac{L}{3}\Bigr)^{+}\right)
-
\phi^{(3)}\!\left(\Bigl(\tfrac{L}{3}\Bigr)^{-}\right)
=\frac{P}{E I},
\]
which is the expected jump in shear.

\textbf{Final model.}
The boundary value problem for the deflection $\phi$ is
\[
\boxed{\;
E I\,\phi^{(4)}(x)=
P\,\delta\!\left(x-\frac{L}{3}\right)+\frac{m g}{L},
\qquad
\phi(0)=\phi'(0)=0,\quad \phi(L)=\phi'(L)=0
\;}
\]
which matches the form requested in the assignment after the change of notation $\phi \leftrightarrow v$ and $\Omega=m g L^{-1}$.

\subsection*{Problem 2}

We study
\[
 -u''(x)+u(x)=\delta\!\Bigl(x-\tfrac12\Bigr),\quad 0<x<1,\quad
 u(0)=u(1)=0 .
\]
Let $L=-\dfrac{d^{2}}{dx^{2}}+1$ on $(0,1)$ with Dirichlet data.
The Green kernel $G(x,a)$ solves $LG(\cdot,a)=\delta(\cdot-a)$ and $G(0,a)=G(1,a)=0$.
Solving $-G''+G=0$ on each side of $x=a$, enforcing continuity of $G$ and the jump rule
\[
G'(a+)-G'(a-)=-1,
\]
gives
\[
G(x,a)=\frac{\sinh(\min\{x,a\})\,\sinh\!\bigl(1-\max\{x,a\}\bigr)}{\sinh 1}.
\]
With the source at $a=\tfrac12$ the solution is
\[
\boxed{\,u(x)=\frac{\sinh\!\bigl(\min\{x,\tfrac12\}\bigr)\,
\sinh\!\bigl(1-\max\{x,\tfrac12\}\bigr)}{\sinh 1}\, } .
\]
Equivalently
\[
u(x)=
\begin{cases}
\dfrac{\sinh x\,\sinh(\tfrac12)}{\sinh 1}, & 0\le x\le \tfrac12,\\[6pt]
\dfrac{\sinh(\tfrac12)\,\sinh(1-x)}{\sinh 1}, & \tfrac12\le x\le 1 .
\end{cases}
\]
The derivative is
\[
u'(x)=
\begin{cases}
\dfrac{\cosh x\,\sinh(\tfrac12)}{\sinh 1}, & 0<x<\tfrac12,\\[6pt]
-\dfrac{\sinh(\tfrac12)\,\cosh(1-x)}{\sinh 1}, & \tfrac12<x<1 .
\end{cases}
\]
Hence \(u\) is continuous and symmetric about \(x=\tfrac12\), while
\[
u'\!\left(\tfrac12+\right)-u'\!\left(\tfrac12-\right)=-1,
\]
so the graph of \(u'\) has a unit negative jump at the source.

\PlotsHeading
\begin{figure}[H]
  \centering
  \incfig{p2_u.png}
  \caption{Exact profile \(u\) for Problem 2}
\end{figure}

\begin{figure}[H]
  \centering
  \incfig{p2_up.png}
  \caption{Derivative \(u'\) with a unit negative jump at \(x=\tfrac12\)}
\end{figure}
\FloatBarrier


\printbibliography
\end{document}
